%% LaTeX2e class for student theses
%% sections/content.tex
%% 
%% Karlsruhe Institute of Technology
%% Institute for Program Structures and Data Organization
%% Chair for Software Design and Quality (SDQ)
%%
%% Dr.-Ing. Erik Burger
%% burger@kit.edu
%%
%% Version 1.3, 2016-12-29

\chapter{Einleitung}
\label{ch:Introduction}

Die Idee dieser Arbeit kommt aus der Wasserrettung. 66\% aller Einsätze des DLRG Landesverbands Württemberg im Wasserrettungsdienst sind Personensuchen im Wasser\cite{dlrgwuerttemberg2017}. Für Gewöhnlich schwimmen Taucher dabei einen Bereich mithilfe von Suchmustern ab. Aufgrund vieler Schwebeteilchen im Wasser ist die Sicht von Tauchern oft unter einen Meter begrenzt, was die Suche verlangsamt. Doch bei einem Ertrinkungsunfall sinkt die Überlebenswahrscheinlichkeit in jeder Minute, die ein Mensch unter Wasser ist \cite{teising2009neonatologische, roeggla2001}.

Daher läuft seit einigen Jahren die Ausstattung der DLRG mit Sonargeräten, die auch bei schlechten Bedingungen einen großen Erfassungsbereich haben. Das beschleunigt die Suche nach vermissten Personen wesentlich. Da das Sonar auf Einsatzbooten angebracht ist, vergeht eine Menge Zeit bis zum Beginn der Suche. Der gesamte Slipvorgang dauert mit Anfahrt der Slipstelle bis zu 15 Minuten.

Ein Weg die Zeit bis zum Einsatz des Sonars zu verkürzen, besteht in der Verwendung eines unbemannten Fahrzeugs. Dieses muss lediglich groß genug sein, um das Sonar zu transportieren. Daher kann es direkt an der Einsatzstelle zu Wasser gelassen werden, was wertvolle Minuten spart.

\section{Example: Citation}
\label{sec:Introduction:Citation}
A citation: \cite{becker2008a} For referencing, see \autoref{sec:Introduction:Figures}

\section{Example: Figures}
\label{sec:Introduction:Figures}
\begin{figure}[h]
\centering
\includegraphics[width=4cm]{logos/sdqlogo}
\caption{SDQ logo}
\label{fig:sdqlogo}
\end{figure}

A reference: The SDQ logo is displayed in \autoref{fig:sdqlogo}. 
(Use \code{\textbackslash autoref\{\}} for easy referencing.) 

\section{Example: Tables}
\label{sec:Introduction:Tables}
\begin{table}[h]
\centering
\begin{tabular}{r l}
\toprule
abc & def\\
ghi & jkl\\
\midrule
123 & 456\\
789 & 0AB\\
\bottomrule
\end{tabular}
\caption{A table}
\label{tab:atable}
\end{table}

\section{Example: Todo-Note}
Meaningless text.
\todo{Replace with meaningful text. (This note is only shown in draft mode.)}

\section{Example: Formula}
One of the nice things about the Linux Libertine font is that it comes with
a math mode package.
\begin{displaymath}
f(x)=\Omega(g(x))\ (x\rightarrow\infty)\;\Leftrightarrow\;
\limsup_{x \to \infty} \left|\frac{f(x)}{g(x)}\right|> 0
\end{displaymath}

%% --------------------
%% | /Example content |
%% --------------------