\documentclass[a4paper]{article}
 
\usepackage[utf8]{inputenc} \usepackage[top=1in, bottom=1.25in, left=1.25in, 
right=1.25in]{geometry}

\DeclareUnicodeCharacter{20AC}{\euro}

\begin{document}

\section{Mechanik}

\subsection{Antrieb}

Im folgenden werden die betrachteten Antriebe vorgestellt. Diese wurden ausgewählt, weil sie die 
Anforderung erfüllen, keine Personen im Wasser zu gefährden.

\subsubsection{Jet-Antrieb}

Es gibt Modellbau-Jetantriebe von Graupner. Diese laufen jedoch gerade aus und sind mechanisch 
sehr aufwendig und benötigen zum Steuern einen Servo. Die Kosten belaufen sich auf ca. 150\euro.

\subsubsection{Propellerantrieb}

Im Internet findet man verschiedene Bootspropeller, die jedoch alle an einen Innenliegenden Motor 
gekoppelt sind.

Es ist jedoch möglich, Brushless-Motoren unter Wasser einzusetzen. BlueRobotics hat ein 
aufeinander abgestimmtes System entwickelt, das Motor, Schraube und Berührungsschutz beinhaltet. 
Dieses Set ist für den Einsatz an Tauchrobotern entwickelt.

\subsection{Steuerung}

Bei der Umsetzung der Steuerung werden die folgenden drei Konzepte miteinander verglichen.

\subsubsection{Rudersteuerung}

Eine Rudersteuerung hat den Nachteil, dass sich das Boot in Bewegung befinden muss, um eine 
Ruderwirkung zu erhalten. Allerdings wäre diese Steuerung am Kostengünstigsten zu realisieren.

\subsubsection{Pod-Antrieb}

Drehbare Propeller ermöglichen das beste Manövrieren. Jedoch ist der Aufbau sehr komplex.

\subsubsection{Differentielle Steuerung}

Diese Steuerung erzeugt durch Verlagerung des Schubs weg von der Kiellinie eine Drehbewegung. Der 
Aufbau ist sehr einfach, da zwei starre Motoren montiert werden, die Kosten sind allerdings sehr 
hoch, da gleich zwei Motoren zum Einsatz kommen.

Die Steuerung läuft in der Software ab.

\subsection{Rumpfform}

Die folgenden Rumpfformen unterscheiden sich in Stabilität, notwendiger Motorisierung und 
möglicher Höchstgeschwindigkeit.

\subsubsection{Verdränger}

Ein Verdränger ist für langsame Fahrten ausgelegt. Er liegt stabil im Wasser und kommt mit sehr 
kleiner Motorisierung aus.

\subsubsection{(Halb-)Gleiter}

(Halb-)Gleiter heben sich ab einer höheren Geschwindigkeit aus dem Wasser. Dadurch wird die 
Reibung vermindert und es ist eine hohe Höchstgeschwindigkeit möglich. Die Stabilität ist jedoch 
eingeschränkt und die Gleitfahrt benötigt eine starke Motorisierung.

\subsubsection{Katamarane}

Katamarane besitzen mehrere kleine Rümpfe, was ihnen sehr hohe Stabilität bei dennoch hohen 
Geschwindigkeiten erlaubt.

\subsubsection{V-Rumpf}

Boote mit V-Rumpf können durch Wellen schneiden, was sie in rauem Wasser überlegen macht. Jedoch 
sehr instabil wenn eine seitliche Kraft wirkt.

vgl http://de.boats.com/dienste/motorboot-ausdruecke-typen-einsatzformen-und-definitionen/

\section{Elektronik}

\subsection{Akku}

\subsubsection{LiPo-Akku}

Lithium-Polymer-Akkus haben eine hohe Energiedichte (130-150 Wh/kg). Jedoch sind sie elektrisch 
und thermisch sehr empfindlich. Das führt soweit, dass bei einer Überladung Brandgefahr besteht.

\subsubsection{LiIon-Akku}

Ein Lithium-Ionen-Akku hat eine ähnliche Leistungsdichte (120-180 Wh/kg) wie der 
Lithium-Polymer-Akku. Dabei ist er weniger anfällig. Der Akku soll bei höchstens 75\% der 
möglichen Ladung gelagert werden, um Alterung zu vermeiden.

\subsubsection{NiMH-Akku}

Nickel-Metallhydrid-Akkus haben eine wesentlich geringere Energiedichte (60-80 Wh/kg) als die 
beiden vorher genannten Typen. Da NiMH-Akkus dem Memory-Effekt unterliegen, müssen sie regelmäßig 
komplett entladen und wieder vollgeladen werden, sonst sinkt ihre Kapazität. Bei der Lagerung 
verliert ein Akku ca 1\% seiner Ladung pro Tag. Eine Ladungserhalung ist aufgrund des 
Memory-Effekts nicht möglich.

\subsubsection{NiCd-Akku}

Nickel-Cadmium-Akkus haben eine geringere Energiedichte (40-50 Wh/kg) als NiMH-Akkus, sind jedoch 
wesentlich zyklenfester. Auf sie trifft auch der Memory-Effekt zu.

\subsubsection{Blei-Akku}

Blei-Akkus sind eine alte und bewährte Technik, sie weisen jedoch die geringste Energiedichte auf 
(30-40 Wh/kg). Sie sind elektrisch und thermisch robust, haben jedoch eine geringe Lebensdauer.

\subsection{Funkübertragung}

Die Funkübertragung soll so ausgelegt sein, dass auch im größten See des Bezirks Kalrsruhe eine Steuerung des Boots von jedem Punkt 
des Ufers aus möglich ist. Der Epplesee hat die größte offene Wasserfläche. Die Entfernung der am weitesten voneinander entfernten 
Ufer beträgt 950m. Als Richtwert für die Funkübertragung soll daher gelten, dass eine Übertragung im Freien über 1km möglich ist.

\subsubsection{Generische 433 MHz Übertragung}

433 MHz Sender und Empfänger sind ein Massenprodukt und daher schon für wenige Euro erhältlich. Die Reichweite dieser Produkte ist 
aber auf wenige 10 Meter begrenzt. Um eine höhere Reichweite zu erhalten, muss ein teureres Modul gekauft werden, das dann eine 
bessere Antenne und hochwertigere HF-Verstärker besitzt.

Generische 433 MHz Sender und Empfänger arbeiten mit einer Amplitudenmodulation, dem sogenannten On-Off-Keying. Dabei wird, abhängig 
vom zu sendenden Bit, entweder ein Träger gesendet (volle Amplitude) oder nicht (Amplitude ist 0).

\subsubsection{WLAN}

Dieser nach IEEE 802.11 zertifizierte Standard hat sehr hohe Datendurchsätze, die im optimalen Fall 10Gbit/s übersteigen.

WLAN hat nach Spezifikation eine Reichweite von bis zu 250m im Außenbereich. Ein in der USA angewandter Standard kann sogar eine 
Übertagung über 5000m gewährleisten. Dabei wird lediglich das Frequenzband und die Leistung des Signals geänert, nicht jedoch der 
zeitliche Ablauf. Daher ist es möglich, mit entsprechenden Richtfunkantennen die WLAN-Reichweite zu erhöhen, sodass eine Übertragung 
über einen Kilometer stattfinden kann.

(Quellen aus \url{https://en.wikipedia.org/wiki/IEEE_802.11})

\subsubsection{Mobilfunk}

Das Mobilfunknetz hat vier Bestandteile, die sich in der Art der Verbindung und der Geschwindigkeit unterscheiden. Die Systeme heißen 
GSM, GPRS, UMTS und LTE. Das Handynetz bietet durch die vorhandene Infrastruktur in Deutschland eine sehr hohe Reichweite, jedoch 
hängt diese von der Position der Basisstationen ab. Diese decken bevorzugt Wohngebiete und weniger die unbewohnte Landschaft ab, 
weshalb die Abdeckung an Seen, dem Einsatzgebiet des Boots, nicht gewährleistet ist.

Ist eine Abdeckung mit UMTS oder LTE gewährleistet, reicht die Übertragungsbandbreite aus, um Live-Bilddaten zu übertragen.

Da man auf eine vorhandene Infrastruktur zurückgreift, muss man hier auch Kosten an den Provider entrichten.

\subsubsection{LoRa}

LoRa wurde speziell für die Übertragung von wenigen Daten über sehr hohe Entfernungen entwickelt. Es unterstützt eine maximale 
Übertragungsrate von 50 kbps. Tests mit verschiedenen LoRa-Modulen haben Übertragungen über mehr als 2 km im urbanen und mehr als 20 
km im ländlichen Gebiet bestätigt.

\section{Software}

\subsection{Betriebssystem}

Ein Betriebssystem abstrahiert die Hardware eines Mikrocontrollers und stellt einheitliche Schnittstellen für darauf laufende Software 
zur Verfügung. Dadurch wird dem Programmierer Arbeit erspart, denn er muss sich nicht in eine neue Hardware einarbeiten. Es stehen 
verschiedene Betriebssysteme zur Verfügung, die gewisse Vor- und Nachteile aufweisen.

Betriebssysteme stellen Services wie Multitasking und Speichermanagement bereit.

\subsubsection{Echtzeitbetriebssysteme}

Echtzeitbetriebssysteme bearbeiten Anfragen immer innerhalb einer im Voraus bestimmten Frist. Das ist für Echtzeitanwendungen wichtig, 
die zum Beispiel regelmäßig eine Berechnung durchführen müssen. Da Echtzeitbetriebssysteme auf Mikrocontroller ausgelegt sind, 
benötigen sie meist wenig Speicher.

\subsection{Baremetal}

Unter Baremetal-Programmierung versteht man die direkte Programmierung des Mikrocontrollers ohne die Verwendung eines Betriebssystems. 
Dabei können alle Features der Hardware voll ausgenutzt werden. Nachteilig ist der wesentlich höhere Aufwand, da eine Einarbeitung in 
die Hardware nötig ist.

\end{document}
