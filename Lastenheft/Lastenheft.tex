\documentclass[a4paper]{article}
 
\usepackage[utf8]{inputenc}
\usepackage[top=1in, bottom=1.25in, left=1.25in, right=1.25in]{geometry}

\begin{document}

\section{Zielbestimmung}

Das Sonarboot soll zur Unterstützung von Wasserrettungsorganisationen eingesetzt werden.
Diese nutzen Sonar bereits um Gewässer nach vermissten Personen oder Gegenständen abzusuchen.
Jedoch ist dazu der Einsatz von Motorrettungsbooten nötig, auf die ein Sonargerät montiert ist.
Dies hat die Nachteile, dass durch den Antrieb eines Motorrettungsboots der Einsatz von Tauchern im
Suchgebiet nict möglich ist und dass das Zuwasserlassen der Boote zeitaufwendig und nicht überall 
möglich ist.

Das Sonarboot soll diese Probleme lösen, indem es die Größe eines Modells hat und über einen für
Taucher ungefährlichen Antrieb verfügt.

\subsection{Musskriterien}

\begin{itemize}
	\item Das Sonarboot muss für Taucher oder andere umstehende Personen ungefährlich sein
\end{itemize}

\subsection{Wunschkriterien}

\begin{itemize}
	\item Die Bilder des Sonarboots sollen über eine kabellose Verbindung an Land übertragen werden
	\item Das Sonarboot soll unsinkbar sein
	\item Das Sonarboot soll definierte Muster automatisch abfahren können
	\item Das Sonarboot soll bei Bedarf manuell gesteuert werden können
\end{itemize}

\subsection{Abgrenzungskriterien}

\begin{itemize}
	\item Die Daten des Sonars sollen nicht analysiert werden, um Personen oder Gegenstände unter 
	Wasser zu erkennen.
\end{itemize}

\section{Produkteinsatz}

\subsection{Anwendungsbereiche}

Das Sonarboot soll in stehenden Gewässern mit bis zu 40m Tiefe zur Aufklärung unter Wasser eingesetzt 
werden. Es ist als Werkzeug für den Primäreinsatz gedacht, da es sehr schnell die Arbeit beginnen kann.

\subsection{Zielgruppen}

Das Sonarboot wird von Einsatzkräften bedient, die nicht besonders darauf geschult sind.

\subsection{Betriebsbedingungen}

Das Sonarboot soll ganzjährig im Wasserrettungsdienst eingesetzt werden können. Die minimale 
Betriebszeit beschränkt sich auf eine Stunde, da danach der Primäreinsatz beendet ist und 
bemannte Motorrettungsboote die Suche fortsetzen können.

\section{Produktumgebung}

\subsection{Software}

\subsection{Hardware}

\subsection{Produktschnittstellen}

\section{Produktfunktionen}

\subsection{Gefährdungsreduktion für Taucher}

Der Antrieb und die Geschwindigkeit des Sonarboots ist so zu wählen, dass selbst bei Kollision/Berührung
einer Person diese nicht zu schaden kommt. Dafür muss der Antrieb so gewählt werden, dass eine innen-
liegende Schraube (Jet-/Impellerantrieb) oder eine mit Propellerschutz ausgestattete, außenliegende
Schraube verwendet wird.

\subsection{Datenübertragung}

Die Daten des Sonarmoduls sollen über eine kabellose Verbindung ans Ufer übertragen werden. Dort können
sie ausgewertet werden.

\subsection{Unsinkbarkeit}

Das Sonarboot soll so gebaut sein, dass selbst bei Beschädigung ein Sinken unmöglich ist. Dazu müssen
Auftriebsmittel als passiver Schutz verwendet werden. Eine aktive Lösung mit aufblasbarem Airbag ist
ebenfalls möglich.

\subsection{Abfahren definierter Muster}

Das Sonarboot soll definierte Muster, die es von einer Basisstation am Ufer bekommt in Form von 
GPS-Koordinaten bekommt, automatisch abfahren können. Dazu muss es seine Position überwachen.
Kollisionen sind so gut wie möglich zu vermeiden. Ebenso soll das Boot nicht in seichtem Wasser 
auflaufen.

\subsection{Manuelle Steuerung}

Bei Bedarf soll das Sonarboot vom automatischen Betrieb auf manuellen Betrieb umgestellt werden können.
Dazu ist eine dauerhafte drahtlose Verbindung aufrechtzuerhalten.

\end{document}
